\section{Physical Attacks}
\label{sec:physical}

It is generally understood that physical access to an x86 platform can
completely compromise software security. Historically, physical
security controls, such as cages, cameras, and locks, have been employed to
prevent or detect physical access. Yet with adoption of
outsourced infrastructure and cloud computing, x86 platforms are
increasingly run outside the physical control of the software owner.

This section briefly summarizes several well-known physical attack vectors
against x86 platforms, including DMA, physical memory extraction, and
platform malware.

\subsection{Direct Memory Access}

By design, x86 architectures provide direct memory access (DMA) from
hardware subsystems to access main memory independently of the
CPU. DMA access is generally used for performance, for example
allowing disk, network, and graphics devices to read and write
data directly to memory without incurring CPU cycles.

Yet, without proper controls, devices with DMA access can read
arbitrary regions of memory and compromise system security by exposing
secrets or allowing an attacker to modify running software in
place. Secrets from captured memory can be extracted easily with
forensics tools like Volatility \cite{Volatility:2014}.

Tribble \cite{Carrier:2004hardware, Grand:2007patent} is an
early example of a device desiged for exfiltrating data via DMA
access, based on an off-the-shelf Intel development kit. Copilot
\cite{Petroni:2004copilot} also used DMA with a PCI device for the
purpose of monitoring kernel integrity. The Maux attack
\cite{Triulzi:2008vd} exploited remote vulnerabilities in a standard
network interface device and accessed memory via DMA. Off-the-shelf
intelligent network adapters, such as those made by Cavium, are able
to exfiltrate DMA-accessed memory over a network connection
\cite{Horovitz:2013physical}.

The IEEE 1394 Firewire interface also provides DMA by design. This led
to several demonstrations of memory extraction to steal data or for
forensics \cite{Dornseif:20040wned, Boileau:2006we,
  Witherden:2010forensics, Dornseif:2005firewire}. The Thunderbolt
interface also offers DMA access and can be exploited in a similar
manner \cite{Maartmann:2011inception}.

\subsection{Physical Memory Extraction}

While DMA may be mitigated by software-based countermeasures,
leveraging a hardware IOMMU such as
Intel VT-d \cite{Intel-IOMMU:2013}, other attacks involve the physical
extraction or modification of system memory. For example, memory bus
analyzer devices are available and can interdict memory
traffic. However, bus analyzers must be installed ahead of time, tend
to be relatively expensive, and physically large.

A ``cold boot'' is a low-cost memory extraction attack that can be
conducted for little cost on a running system
\cite{Halderman:2008tp}. Cold booting involves literally freezing
system memory modules with an aerosol freeze spray. The memory
contents are preserved long enough to boot to a ``scraper'' image such
as {\em bios-memimage} or {\em msramdump} which can preserve the
memory contents to persistent storage.

Cold booting disrupts a running system and data must be recovered
immediately, before the memory module thaws. Furthermore, it does not
reliably capture all memory contents, as there is some degradation
over time. Conducting the attack may be further complicated by
error-correcting memory which is cleared on a reset or by data
scrambling for power supply noise suppression \cite{Mozak:2011lfsr}.

Non-volatile memory (NV-RAM), designed to persist data after a power
loss, is now available in DDR3/DDR4 form-factors used by standard x86
servers. An attacker can install NV-RAM modules in a server and remove
them at any moment, recovering the data at a later time. If a memory
mirroring mode is configured, an attacker could remove a NV-RAM module
from a running system and replace it without disrupting service.


\subsection{Bootkits and Platform Malware}

\begin{itemize}
  \item APT
\end{itemize}

\subsection{Practical Attack Vectors}

{\em SW: This might be unnecessary. Do we really need to explain how people get
physical access? }

{\em CW: Suggest writing a few sentences, and moving to intro for this
section.  The NSA attacks seem particularly interesting to at least mention.}

\begin{itemize}
  \item Supply chain
  \item Datacenter insiders
  \item Previous bare-metal host tenants: Can we pwn our SoftLayer BIOS?
  \item Stoned bootkit in the wild
  \item NSA ANT
\end{itemize}
