\label{sec:abstract}

Defending computers from unauthorized physical access and malicious
hardware devices has proved extremely challenging.  The problem is
exacerbated in cloud-computing environments, where users lack physical
control over the server hardware that executes their workloads.
Moreover, cloud service providers can be compelled by government
agencies to provide physical access to servers.

We introduce {\em \vcage}, a software-based cryptoprocessor system
that employs cryptographic techniques to provide confidentiality for
unmodified workloads. Code and data are stored as cleartext only
within the processor cache, but remain encrypted in main memory.  We
implemented \vcage\ in a commercial product for commodity server
platforms based on Intel x86 processors and industry-standard TPMs.

We present the design and implementation of \vcage\ in Linux, and show
how it virtualizes physical security by protecting data-in-use
transparently for KVM virtual machines.  Quantitative experiments
demonstrate that \vcage\ effectively safeguards privacy, while
achieving acceptable performance for a wide range of workloads.

